\documentclass[11pt]{article}

\usepackage{fullpage,epsfig,latexsym,picinpar,amsbsy,amsmath,algorithm,mathtools}
\usepackage[noend]{algpseudocode}
\usepackage{xspace}

\usepackage{listings}
\usepackage{color}
\lstset{
  basicstyle=\ttfamily,
  mathescape
}

\lstset{frame=tb,
  language=Java,
  aboveskip=3mm,
  belowskip=3mm,
  showstringspaces=false,
  columns=flexible,
  basicstyle={\small\ttfamily},
  numbers=none,
  numberstyle=\tiny\color{gray},
  keywordstyle=\color{blue},
  commentstyle=\color{dkgreen},
  stringstyle=\color{mauve},
  breaklines=true,
  breakatwhitespace=true,
  tabsize=3
}

\begin{document}

\centerline{\large \bf EE232 Final Project Report}
\centerline{Yidi Wang}
\centerline{862114701}

\vskip 0.1in


\paragraph{Summary of Contributions} \mbox{} \\
Nowadays, the use of photovoltaic (PV) is widespread because of the high demand of $CO_2$ emission reduction and renewable energy. However, the power generation from PV is not stable because people cannot control the light intensity from the sun and the ambient temperature. Therefore, how to match the intermittent energy production with the dynamic demand is a major challenge for the use of PV. The solution is to connect a energy storage unit to the network. This integration is currently limited by two constraints: regulations and power flow managment. In this paper, the authors focus on the power flow managment problem, which is about how to optimize the use of storage, the use of the PV source, and to match the local production with the local consumption. They proposed an optimal power managment mechanism for grid connected photovoltaic (PV) systems with storage. 



\paragraph{Critical Review} \mbox{} \\


\paragraph{Improvement} \mbox{} \\



\end{document}

